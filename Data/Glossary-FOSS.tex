% Terms that need definition:
%
% (Students: You *cannot* use these terms before they have been referenced in the course! :-)
% license compliance
% license compatibility,
% inbound/outbound license
% four freedoms of software
% share-alike/reciprocal/copyleft/permissive/academic
% viral effect
% tivoization
% freedom to operate (ability to exercise 4 freedoms and to pursue own interest)
% automatic licensing
% transitive licensing
% non-negotiable licensing
% enforcement of licensing terms

\newacronym[see={FOSS}]{FLOSS}{FLOSS}{free/libre and open source software}
\newacronym[see={FOSS}]{OSS}{OSS}{open source software}
\newacronym{GA}{GA}{general assembly}

\newdualentry{FOSS}{FOSS}{free and open source software}{The term free
  and open source software refers to software that is distributed
  under a license which complies with the \gls{OSD}. The \gls{OSI} is
  the steward that approves licenses for being compliant with this
  definition}

\newglossaryentry{KDE}{name={KDE}, description={KDE is a recursive
    acronym that refers to the ``K Desktop Environment'', a graphical
    computing environment for free software operating systems. It is
    produced by the KDE community\footnote{\url{https://www.kde.org}},
    a global, volunteer driven and decentralized \gls{FOSS} community}
}

\newglossaryentry{governance}{name={governance}, description={In the
    context of \gls{FOSS}, governance describes the totality of
    implicit and explicit behavioral norms, codes and processes that
    regulate the relationship between contributors and the community
    as a whole} }

\newglossaryentry{CoC}{name={code of conduct}, description={A code of
    conduct describes the behavioral norms and rules that community
    contributors are expected to adhere to. Once adopted, a code of
    conduct is part of the explicit \gls{governance} norms of a community} }

\newglossaryentry{osw}{name={open source way}, description={The
    colloquial term ``the open source way'' describes the mutual
    understanding of \gls{SF}, values and principles applied by the
    \gls{WOSC}, especially the norms of non-negotiable \gls{OS}
    licenses, open governance and \gls{meritocracy}}}

\newglossaryentry{meritocracy}{name={meritocracy}, description={In the
    context of \gls{FOSS}, the term meritocracy is used to describe a
    system where contributors gain reputation in a community solely
    based on the value of the contributions they make} }

\newglossaryentry{WOSC}{name={wider Open Source community}, description={The
  phrase \emph{wider Open Source community} is commonly used to describe the
  totality of individuals, smaller and larger communities, umbrella
  organizations and entities collaborating on developing the commons of
  \gls{FOSS} in the global \gls{UDM}} }

\newglossaryentry{OSC}{name={community}, description={A \gls{FOSS}
    community produces information goods, predominantly computer
    programs, in a collaborative process based on voluntary
    participation} }

\newglossaryentry{osc}{
  name={open source culture},
  description={A synonym for the \gls{osw}}
}

\newglossaryentry{LD}{ name={Linux distribution}, description={A Linux
    distribution is a software collection based on Linux that aggregates the
    work of the \gls{WOSC} into a complete operating system. Linux distributions
    are an integral part of the \gls{UDM} because they make \gls{FS} accessible
    for specific applications like end-user desktops or embedded systems} }

\newglossaryentry{UDM}{name={upstream/downstream model}, description={The
  analogy of the upstream/downstream model uses the mental image of a large
  river that collects the water from many smaller and smaller tributaries (the
  communities) and delivers it to the ocean (the users). Key tenets of the
  upstream/downstream model are the non-negotiability of the \gls{FS} licensing
  terms and community governance norms} }

\newglossaryentry{LF}{ name={Linux Foundation}, description={The Linux
    Foundation is a not-for-profit industry association (US 501(c)(6)) dedicated
    to building sustainable ecosystems around open source projects to accelerate
    technology development and commercial adoption}}

\newglossaryentry{CWG}{name={community working group},
  description={The community working group of KDE is the only formal
    conflict resolution mechanism the KDE community has in place. ``The
    long-term goal of the Community Working Group is to help to
    maintain a friendly and welcoming KDE community, thereby ensuring
    KDE remains a great project enjoyed by all contributors and
    users.''\footnote{\url{https://ev.kde.org/workinggroups/cwg.php}} } }

\newglossaryentry{EF}{ name={Eclipse Foundation}, description={The Eclipse
    Foundation is a not-for-profit industry association (US 501(c)(6)) that
    provides a global community of individuals and organizations with a mature,
    scalable, and commercially focused environment for collaboration and
    innovation} }

\newglossaryentry{eV}{name={KDE~e.V.}, description={KDE
    e.V.\footnote{\url{https://ev.kde.org/}} is a non-profit organization
    registered in Berlin, Germany that represents the KDE Community in
    legal and financial matters } }

\newglossaryentry{FSFE}{name={FSFE}, description={The Free Software
    Foundation Europe ``is a charity that empowers users to control
    technology.''\footnote{\url{https://fsfe.org/about/about.en.html}}} }

\newglossaryentry{FSF}{name={FSF}, description={``The Free Software
    Foundation (FSF) is a nonprofit with a worldwide mission to
    promote computer user freedom.''\footnote{\url{http://www.fsf.org/}}} }

\newglossaryentry{FS}{name={free software}, description={Free software
    is made available to everybody under a license that gives the user
    the freedom to run, copy, distribute, study, change and improve
    the software. ``'free software' is a matter of liberty, not
    price''.\cite{RMS_essays_2010} The \gls{FSF} and the \gls{OSI}
    maintain lists of
    licenses\footnote{\url{https://www.gnu.org/licenses/license-list.html},
      \url{https://opensource.org/licenses}} that provide those ``four
    essential freedoms'' to recipients of the software. The terms of
    all licenses that provide these terms turn the product into a
    common good, and also ensure that the product itself will continue
    to be freely licensed. This effect may or may not extend to
    derivative works, resulting in the classification of the licenses
    into reciprocal and permissive categories} }

\newglossaryentry{FTF}{name={Freedom Task Force}, description={The
    {\em Freedom Task Force} is an intiative of \gls{FSFE} that
    ``seeks to help programmers properly set up and organise projects
    legally, as well as educate companies to understand how the GPL
    works.  As needed, the purpose of the group will also include
    enforcement in the case of licence
    violations.''\footnote{\url{http://mail.fsfeurope.org/pipermail/press-release/2006q4/000159.html}}
    The Freedom Task Force was created in 2006 } }

\newglossaryentry{LN}{name={Legal Network}, description={The
    \gls{FSFE} {\em Legal Network}, initially called the European
    Legal Network, is a ``neutral, non-partisan, group of experts in
    different fields involved in Free Software legal issues. Currently
    the Legal Network has over 400 participants from different legal
    systems, academic backgrounds and
    affiliations.''\footnote{\url{https://fsfe.org/activities/ftf/ln.en.html}}
} }

\newglossaryentry{OS}{name={open source}, description={The term ``open
    source'' is used in this paper as a synonym to \gls{FOSS}. It
    originally describes a campaign to promote \gls{FS} to
    business.\cite{perens-2017}}, see={FS} }

\newglossaryentry{OSI}{name={Open Source Initiative}, description={The
    Open Source Initiative\footnote{\url{https://opensource.org}} was
    founded in 1998 by Eric Raymond and Bruce Perens. It is dedicated
    to the promotion of open source software. The term ``open source''
    was coined by the initiative's founders. It created the initial
    \gls{OSD}. Today the Open Source Initiative is the steward
    stewards of the Open Source Definition (OSD) and the
    community-recognized body for reviewing and approving licenses as
    OSD-conformant. It publishes the list of all approved open source
    licenses.\footnote{\url{https://opensource.org/licenses/alphabetical}}
    The The Open Source Initiative is a California public benefit
    corporation, with 501(c)3 tax-exempt status} }

\newglossaryentry{OSD}{name={Open Source Definition}, description={The Open
    Source Definition formulates the terms software must comply with to be
    considered \gls{FOSS}.\footnote{\url{https://opensource.org/osd}} It is
    maintained by the \gls{OSI}. By way of the Open Source Definition,
    especially it's unanimous acceptance, the term ``open source'' gained a
    precise meaning across the wider \gls{FOSS} community, and is therefore a
    term of art and part of open source culture}}

\newglossaryentry{SF}{name={software freedom}, description={``Software
    freedom'' is what distributing a work under a \gls{FLOSS} license
    affords the user. While the terms \gls{FS} and \gls{OS} are used
    mostly synonymously today, ``software freedom'' is a political
    goal elements of the \gls{FS} movement aim for. It can be
    circumscribed as the absence of coercion to use proprietary
    software.\cite{RMS_essays_2010} Organisations like the Software
    Freedom Conservancy\footnote{\url{https://sfconservancy.org/}} and the
    Software Freedom Law
    Center\footnote{\url{https://www.softwarefreedom.org/}} work to advance
    software freedom} }

\newglossaryentry{WM}{name={Wikimedia}, description={Wikimedia is the
    global community that creates \gls{WP}}}

\newglossaryentry{WMF}{name={Wikimedia Foundation}, description={The
    Wikimedia Foundation is a US-based non-profit
    organisation that represents Wikipedia globally} }

\newglossaryentry{WMDE}{name={Wikimedia Deutschland e.V.},
  description={Wikimedia Deutschland e.V.  is a non-profit
    organization registered in Berlin, Germany that represents the
    German language Wikipedia community } }

\newglossaryentry{WP}{name={Wikipedia}, description={Wikipedia is an
    online encyclopedia that aims at making the knowledge of the world
    available to everybody} }

\newglossaryentry{VS}{name={versioning}, description={Versioning of
software is the unique naming of software identifying it's development.
For example, version 1.0 is commonly used to identify a first edition.} }

\newglossaryentry{DSC}{name={debian social contract},
  description={``Debian social contract'' refers to Debian systems'
    commitment towards the \gls{FS} community. The DBS is a 'moral
    agenda' which is based on the Debian Free Software Guidlines
    according to which Debian promises to: be entirely free; provide
    the best work; will not try to cover problems; give priority to
    users and free software; make it possible for the software to be
    used with non-free
    software\footnote{\url{https://www.debian.org/social_contract}} }
}

\newglossaryentry{DRM}{name={DRM}, description={Digital Rights
    Management (DRM) is a collection of technologies that can be used
    to protect the copyright of electronic media including software
    and multimedia content. The way it is done mostly includes
    restricting the use of hardware and copyrighted works.  For
    example, certain multimedia companies can restrict the number of
    devices the media can be played on or force customers to use a
    specific application / device to play the contents of the desired
    media.
    \footnote{\url{https://books.google.com/books/about/Computer_Forensics_Investigating_Network.html?hl=de&id=0rVfRwcIPYgC}}
    \footnote{\url{https://www.eff.org/issues/drm}}
  } }

%%% Removed as not FOSS related:
%% \newglossaryentry{RE}{name={Reverse Engineering}, description={Reverse
%%     engineering is taking apart an object to see how it works in order
%%     to duplicate or enhance the object.  The practice, taken from
%%     older industries, is now frequently used on computer hardware and
%%     software. Software reverse engineering involves reversing a
%%     program's machine code (the string of 0s and 1s that are sent to
%%     the logic processor) back into the source code that it was written
%%     in, using program language statements.
%% \footnote{\url{https://en.wikipedia.org/wiki/Reverse_engineering}}
%% \footnote{\url{https://searchsoftwarequality.techtarget.com/definition/reverse-engineering}}
%% } }

\newglossaryentry{GIT}{name={Git}, description={Git is open source
    software used for distributed versioning of files.  It was
    invented in 2005 by Linus Torvald, who is also the creator of the
    Linux kernel} }

\newglossaryentry{SC}{name={source code}, description={Source code is
    the fundamental component of a computer program that is created by
    a programmer and is referred to as the "before" versions of a
    compiled computer
    program\footnote{\url{https://searchmicroservices.techtarget.com/definition/source-code}}
} }

%%% Commented because the description is incorrect:
%% \newglossaryentry{CM}{name={committer}, description={Committers are
%%     the community's quality control experts. Somebody has to control
%%     the process.  Committers actually decide what changes, based on
%%     community experience, make it into the originally licensed
%%     program.  A committer is an individual who is able to modify the
%%     source code of a particular piece of open-source software.
%%    \footnote{\url{https://www.techrepublic.com/blog/10-things/10-terms-and-concepts-related-to-open-source/}}
%%  }
%% }

\newglossaryentry{CND}{name={continuous-non-differentiating
    cooperation}, description={Continuous-non-differentiating
    cooperation is a collaboration model implemented in \gls{FOSS}
    communities and facilitated by \gls{FOSS} foundations that enables
    otherwise competing market actors to continuously cooperate to
    develop a common software stack that serves as basic,
    non-differentiating technology prerequisite to products that
    combine free and proprietary software. In contrast to
    pre-competitive cooperation on the development of proprietary,
    differentiating products, anti-trust concerns are not relevant in
    the continous-non-differentiating cooperation model since
    collusion is impossible if the results are immediately available
    to the general public and the development process is generally
    open for participation and forking by all interested stakeholders}
}

\newglossaryentry{ICT}{name={ICT}, description={``Information and
    communication technology'' describes the computing and
    telecommunications industry sector. The sector serves the
    information processing, storage and networking demand of the
    digital economy } }

\newglossaryentry{OI}{name={open innovation}, description={A business
    concept which encourages companies to acquire outside sources of
    innovation in order to improve product lines and shorten the time
    required to bring products to market or to release internally
    developed
    innovation\footnote{\url{http://www.businessdictionary.com/definition/open-innovation.html}}
} }

\newglossaryentry{physicalgoods}{name={physical goods},
  description={Physical goods require at least a combination of labor
    and materials to build. Assuming marginal cost pricing, the price
    of a physical good is affected by the factors required for its
    production, and therefore larger than zero} }

\newglossaryentry{infogoods}{name={information goods},
  description={Information goods are expensive to create, but only
    incur negligible cost of reproduction. Assuming marginal cost
    pricing, the price of an information good converges to zero under
    perfect competition} }

\newglossaryentry{sstack}{name={software stack}, description={Computer
    systems consist of multiple layers of subsystems that together
    provide a platform that applications run on. Lower parts of the
    stack are usually less differentiating and typically
    \gls{FOSS}. Some stacks are so common that they have names, like
    the LAMP (Linux, Apache, MySQL, PHP) stack} }

\newglossaryentry{commodity}{name={commodity}, description={A
    commodity is a good with normalized attributes, giving instances
    of it fungibility. Commodities commonly are produced under price
    competition. \Gls{FOSS} that implements common,
    non-differentiating features of a computer system is considered a
    commodity} }

\newglossaryentry{reciprocial}{name={reciprocial public license}, description={The 
    reciprocial public license is a copyleft software license. This License is 
    based on the concept of reciprocity. In exchange for being granted certain 
    rights under the terms of this License to Licensor's Software, whose Source 
    Code You have access to, You are required to reciprocate by providing equal 
    access and rights to all third parties to the Source Code of any Modifications.} }