% Terms that need definition:
%
% (Students: You *cannot* use these terms before they have been referenced in the course! :-)
% license compliance
% license compatibility,
% inbound/outbound license
% governance
% four freedoms of software
% share-alike/reciprocal/copyleft/permissive/academic
% viral effect
% tivoization
% FLOSS/FOSS/OSS
% freedom to operate (ability to exercise 4 freedoms and to pursue own interest)
% automatic licensing
% transitive licensing
% enforcement of licensing terms

\newacronym[see={FOSS}]{FLOSS}{FLOSS}{free/libre and open source software}
\newacronym{FOSS}{FOSS}{free and open source software}
\newacronym{GA}{GA}{general assembly}

\newglossaryentry{KDE}{name={KDE}, description={KDE is a recursive
    acronym that refers to the ``K Desktop Environment'', a graphical
    computing environment for free software operating systems. It is
    produced by the KDE community, a global, volunteer driven and
    decentralised \gls{FLOSS} community} }

\newglossaryentry{CoC}{name={code of conduct}, description={A code of
    conduct describes the behavioral norms and rules that community
    contributors are expected to adhere to } }

\newglossaryentry{CWG}{name={community working group},
  description={The community working group of KDE is the only formal
    conflict resolution mechanism the community has in place. ``The
    long-term goal of the Community Working Group is to help to
    maintain a friendly and welcoming KDE community, thereby ensuring
    KDE remains a great project enjoyed by all contributors and
    users.''\footnote{https://ev.kde.org/workinggroups/cwg.php} } }

\newglossaryentry{eV}{name={KDE e.V.}, description={KDE
    e.V.\footnote{https://ev.kde.org/} is a non-profit organization
    registered in Berlin, Germany that represents the KDE Community in
    legal and financial matters } }

\newglossaryentry{FSFE}{name={FSFE}, description={The Free Software
    Foundation Europe ``is a charity that empowers users to control
    technology.''\footnote{https://fsfe.org/about/about.en.html}} }

\newglossaryentry{FSF}{name={FSF}, description={``The Free Software
    Foundation (FSF) is a nonprofit with a worldwide mission to
    promote computer user freedom.''\footnote{http://www.fsf.org/}} }

\newglossaryentry{FS}{name={free software}, description={Free software
    is made available to everybody under a license that gives the user
    the freedom to run, copy, distribute, study, change and improve
    the software. ``'free software' is a matter of liberty, not
    price''.\cite{RMS_essays_2010} The \gls{FSF} and the \gls{OSI}
    maintain lists of
    licenses\footnote{https://www.gnu.org/licenses/license-list.html,
      https://opensource.org/licenses} that provide those ``four
    essential freedoms'' to recipients of the software. The terms of
    all licenses that provide these terms turn the product into a
    common good, and also ensure that the product itself will continue
    to be freely licensed. This effect may or may not extend to
    derivative works, resulting in the classification of the licenses
    into reciprocal and permissive categories} }

\newglossaryentry{FTF}{name={Freedom Task Force}, description={The
    {\em Freedom Task Force} is an intiative of \gls{FSFE} that
    ``seeks to help programmers properly set up and organise projects
    legally, as well as educate companies to understand how the GPL
    works.  As needed, the purpose of the group will also include
    enforcement in the case of licence
    violations.''\footnote{http://mail.fsfeurope.org/pipermail/press-release/2006q4/000159.html}
    The Freedom Task Force was created in 2006 } }

\newglossaryentry{LN}{name={Legal Network}, description={The
    \gls{FSFE} {\em Legal Network}, initially called the European
    Legal Network, is a ``neutral, non-partisan, group of experts in
    different fields involved in Free Software legal issues. Currently
    the Legal Network has over 400 participants from different legal
    systems, academic backgrounds and
    affiliations.''\footnote{https://fsfe.org/activities/ftf/ln.en.html}
} }

\newglossaryentry{OS}{name={open source}, description={The term ``open
    source'' is used in this paper as a synonym to \gls{FS}}, see={FS} }

\newglossaryentry{OSI}{name={Open Source Initiative}, description={The
    Open Source Initiative\footnote{https://opensource.org} was
    founded in 1998 by Eric Raymond and Bruce Perens. It is dedicated
    to the promotion of open source software. The term ``open source''
    was coined by the initiative's founders. It created the initial
    \gls{OSD}. Today the Open Source Initiative is the steward
    stewards of the Open Source Definition (OSD) and the
    community-recognized body for reviewing and approving licenses as
    OSD-conformant. It publishes the list of all approved open source
    licenses.\footnote{\url{https://opensource.org/licenses/alphabetical}}
    The The Open Source Initiative is a California public benefit
    corporation, with 501(c)3 tax-exempt status} }

\newglossaryentry{OSD}{name={Open Source Definition}, description={The
    Open Source Definition formulates the terms software must comply
    with to be considered
    \gls{FLOSS}.\footnote{\url{https://opensource.org/osd}} It is
    maintained by the \gls{OSI}. By way of the Open Source Definition,
    especially it's unanimous acceptance, the term ``open source''
    gained a precise meaning across the wider \gls{FLOSS} community,
    and is therefore a term of art and part of open source culture}
}

\newglossaryentry{SF}{name={software freedom}, description={``Software
    freedom'' is what distributing a work under a \gls{FLOSS} license
    affords the user. While the terms \gls{FS} and \gls{OS} are used
    mostly synonymously today, ``software freedom'' is a political
    goal elements of the \gls{FS} movement aim for. It can be
    circumscribed as the absence of coercion to use proprietary
    software.\cite{RMS_essays_2010} Organisations like the Software
    Freedom Conservancy\footnote{https://sfconservancy.org/} and the
    Software Freedom Law
    Center\footnote{https://www.softwarefreedom.org/} work to advance
    software freedom} }

\newglossaryentry{WM}{name={Wikimedia}, description={Wikimedia is the
    global community that creates \gls{WP}}}

\newglossaryentry{WMF}{name={Wikimedia Foundation}, description={The
    Wikimedia Foundation is a San Franscisco based non-profit
    organisation that represents Wikipedia globally\todo{improve} } }

\newglossaryentry{WMDE}{name={Wikimedia Deutschland e.V.},
  description={Wikimedia Deutschland e.V.  is a non-profit
    organization registered in Berlin, Germany that represents the
    German language Wikipedia\todo{improve} } }

\newglossaryentry{WP}{name={Wikipedia}, description={Wikipedia is an
    online encyclopedia that aims at making the knowledge of the world
    available to everybody\todo{improve} } }

\newglossaryentry{VS}{name={versioning}, description={Versioning of 
software is the unique naming of software identifying it's development. 
For example, version 1.0 is commonly used to identify a first edition.} }

\newglossaryentry{DSC}{name={debain social contract}, description={``Debian 
    social contract'' refers to Debian systems' commitment towards the free 
    software community. The DBS is their 'moral agenda' which is based on the 
    Debian Free Software Guidlines according to which Debian promises to: by 
    entirely free;provide the best work; will not try to cover problems; give 
    priority to users and free software; make iot possible for the software to 
    be used with non-free software\footnote{\url{https://www.debian.org/social_contract/}} } }

\newglossaryentry{DRM}{name={DRM}, description={Digital Rights Management
    (DRM) is a collection of technologies that can be used to protect the copyright
    of electronic media including software and multimedia content. The way it
    is done mostly includes restricting the use of hardware and copyrighted works.
    For example, certain multimedia companies can restrict the number of devices the
    media can be played on or force customers to use a specific application / device
    to play the contents of the desired media.
    \footnote{https://books.google.com/books/about/Computer\_Forensics\_Investigating\_Network.html?hl=de\&id=0rVfRwcIPYgC}
    \footnote{https://www.eff.org/issues/drm}
  } }

\newglossaryentry{RE}{name={Reverse Engineering}, description={Reverse engineering is 
taking apart an object to see how it works in order to duplicate or enhance the object. 
The practice, taken from older industries, is now frequently used on computer hardware and software. Software reverse engineering involves reversing a program's machine code (the string of 0s and 1s that are sent to the logic processor) back into the source code that it was written in, using program language statements.
\footnote{\url{https://en.wikipedia.org/wiki/Reverse_engineering}}
\footnote{\url{https://searchsoftwarequality.techtarget.com/definition/reverse-engineering}}
} }
