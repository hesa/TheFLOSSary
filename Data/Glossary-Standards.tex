%%% A common glossary for standards and standard setting related terms.
%%% (C) 2018, Mirko Boehm
%%% License: CC-BY-SA 2.0
%%%
%%% For terms that are also acronyms, always reference the acronyms in
%%% the document. Using the acronym will automatically reference the
%%% glossary entry (if the acronym definition contains a \glsadd
%%% statement).

% Terms that need definition:
% formal/informal standards,
% FRAND (and transparency of terms),
% interoperability, disclosure (patents), procurement, mandating, SEP

%%% See https://en.wikibooks.org/wiki/LaTeX/Glossary for syntax and
%%% semantics.
\usepackage{xparse}
%%% A dual entry is an acronym that is also a defined term. It has an
%%% acronyms entry and a glossary entry that cross-reference each
%%% other.

%%% Acronyms:
\newacronym[see={fsry:SDO}]{fsry:SSO}{SSO}{standards setting organization}
\newacronym[see={fsry:FRAND}]{fsry:RAND}{RAND}{reasonable and non-discriminatory}

%%% Entries:
\newdualentry{fsry:SDO}{SDO}{standards development organization}{A
  standards development organization is an entity whose primary
  activities are developing, coordinating, promulgating, revising,
  amending, reissuing, interpreting or otherwise maintaining standards
  that address the interests of a wide base of users.}

\newdualentry{fsry:FRAND}{FRAND}{fair, reasonable and
  non-discriminatory}{Licensing under fair, reasonable and
  non-discriminatory terms is a voluntary commitment some \gls{fsry:SDO}
  request from a patent owner that participates in standards
  development.}

\newdualentry{fsry:SEP}{SEP}{standards essential patent}{A standard
  essential patent claims an invention that must be used to comply
  with a technical standard. \gls{fsry:SSO} establish policies that
  regulate towards their participants how standard essential patents
  are expected to be licensed.}

\newdualentry{fsry:ISO}{ISO}{International Organization
  for Standardization}{The International Organization
  for Standardization is an international, independent,
  non-governmental standard setting body that consists of delegates
  from \glspl{fsry:NSB}, with headquarters in Geneva, Switzerland.}

\newdualentry{fsry:W3C}{W3C}{World Wide Web Consortium}{ The World Wide Web
  Consortium is an international community that develops open
  standards to ensure the long-term growth of the Web.}

\newglossaryentry{fsry:NSB} { name = {national standards body}, plural =
  {national standards bodies}, description = {Countries nominate one
    \gls{fsry:SSO} that represents them in \gls{fsry:ISO}. These appointed
    \glspl{fsry:SSO} are referred to as national standards bodies. They
    often have a special privileged relationship with their
    country. For example, a treaty between the federal republic of
    Germany and DIN declares DIN to be the national standards body of
    Germany, and in turn obligates DIN to the public benefit, among
    other terms.}}

\newglossaryentry{fsry:std} { name = {standard},
  description = {A standard is a document established by consensus and
    approved by a recognized body, that provides, for common and
    repeated use, rules, guidelines or characteristics for activities
    or their results, aimed at the achievement of the optimum degree
    of order in a given context (EN 45020:2006).} }

\newglossaryentry{fsry:standardization} { name = {standardization},
  description = {Standardization describes an activity of
    establishing, with regard to actual or potential problems,
    provision for common and repeated use, aimed at the achievement of
    the optimum degree of order in a given context (EN 45020:2006).} }

\newdualentry{fsry:IP}{IP}{intellectual property}{Intellectual property
  (IP) is a term that describes intangible creations of the human
  intellect that can be controlled by an owning entity, like artistic
  works, inventions and designs. IP is made a tradeable good through
  the application of \gls{fsry:IPR}. }

\newdualentry{fsry:IPR}{IPR}{intellectual property rights}{Intellectual
  property rights (IPR) include copyright, designs, patents,
  trademarks and other rights that are associated with \gls{fsry:IP} and
  are utilized to grant permission to use the work through licensing
  and other relationships.}

\newglossaryentry{fsry:SPEC} { name = {specification}, description = {In
    the context of \gls{fsry:standardization}, the term specification
    refers to a set of documents that describe the requirements to be
    satisfied by a technical standard.} }

\newglossaryentry{fsry:IMPL} { name = {implementation}, description = {In
    the context of \gls{fsry:standardization}, the term implementation
    refers to a product that is compliant with the specification of a
    standard.} }

\newglossaryentry{fsry:SE} { name = {standardizing effect}, description = {
    A standardizing effect is observable in the adoption of a common
    technical solution that results from the application of a
    \gls{fsry:SI}.} }

\newglossaryentry{fsry:SI} { name = {standardization instrument},
  description = { A standardization instrument is a mechanism applied
    by stakeholder that causes a \gls{fsry:SE}. Examples for
    standardization instruments are recognized \gls{fsry:SSO}, normalized
    customs and practices enforced by tradition, codes of behavior
    that are prevalent in some industry sectors, especially trade, but
    also industrial consortia, professional charters, or \gls{fsry:FOSS}
    governance.} }

\newglossaryentry{fsry:OpenStd} { name = {open standard}, description = {
    The European Interoperability Framework requires for open
    standards to give all stakeholders the opportunity to contribute
    to the development of the specification, the availability of the
    specification to everybody to study, and for the relevant
    \gls{fsry:IPR} to be licensed ``on \gls{fsry:FRAND} terms, in a way that
    allows implementation in both proprietary and open source
    software, and preferably on a royalty-free
    basis''.\cite{flossary:ec-sep-2018} } }
