%%% A common glossary for standards and standard setting related terms.
%%% (C) 2018, Mirko Boehm
%%% License: CC-BY-SA 2.0
%%%
%%% For terms that are also acronyms, always reference the acronyms in
%%% the document. Using the acronym will automatically reference the
%%% glossary entry (if the acronym definition contains a \glsadd
%%% statement).

% Terms that need definition:
% formal/informal standards, FRAND (and transparency of terms),
% interoperability, disclosure (patents), procurement, mandating, SEP

%%% See https://en.wikibooks.org/wiki/LaTeX/Glossary for syntax and
%%% semantics.
\usepackage{xparse}
%%% A dual entry is an acronym that is also a defined term. It has an
%%% acronyms entry and a glossary entry that cross-reference each
%%% other.

%%% Acronyms:
\newacronym[see={SDO}]{SSO}{SSO}{standards setting organization}

%%% Entries:
\newdualentry{SDO}{SDO}{standards development organization}{A
  standards development organization is an entity whose primary
  activities are developing, coordinating, promulgating, revising,
  amending, reissuing, interpreting or otherwise maintaining standards
  that address the interests of a wide base of users}

\newdualentry{ISO}{ISO}{International Organization
  for Standardization}{The International Organization
  for Standardization is an international, independent,
  non-governmental standard setting body that consists of delegates
  from \glspl{NSB}, with headquarters in Geneva, Switzerland}

\newglossaryentry{NSB} { name = {national standards body}, plural =
  {national standards bodies}, description = {Countries nominate one
    \gls{SSO} that represents them in \gls{ISO}. These appointed
    \glspl{SSO} are referred to as national standards bodies. They
    often have a special privileged relationship with their
    country. For example, a treaty between the federal republic of
    Germany and DIN declares DIN to be the national standards body of
    Germany, and in turn obligates DIN to the public benefit, among
    other terms}}

\newglossaryentry{std} { name = {standard}, plural = {standards},
  description = {a document established by consensus and approved by a
    recognized body, that provides, for common and repeated use,
    rules, guidelines or characteristics for activities or their
    results, aimed at the achievement of the optimum degree of order
    in a given context (EN 45020:2006)} }

