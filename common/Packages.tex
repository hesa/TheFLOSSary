\usepackage{a4}
\usepackage[utf8]{inputenc}
\usepackage[T1]{fontenc}
\usepackage[hyphens]{url}
\usepackage{breakurl}
\usepackage[breaklinks]{hyperref}
\usepackage{chronology}
\usepackage{array}
\ifdefined\enabletodos
\usepackage[colorinlistoftodos, textsize=small]{todonotes}
\else
\usepackage[disable]{todonotes}
\fi
\reversemarginpar
\usepackage{placeins}
\usepackage{soul}
\usepackage{MnSymbol,wasysym}
\usepackage{ntheorem}
\theoremseparator{:}
\newtheorem{hyp}{Hypothesis}
\usepackage{acronym}
\usepackage[toc,xindy,acronym,nopostdot]{glossaries}
\usepackage{ccicons}
\usepackage{lipsum}
\usepackage[backend=biber,style=numeric,citestyle=numeric,doi=false,isbn=false,eprint=false]{biblatex}

%%% Use this as \newdualentry{symbol}{acronym}{full text}{description}:
\providecommand\dualfsryentry{}
\renewcommand*{\dualfsryentry}[5][]{%
  \newglossaryentry{fsry:main-#2}{name={#4},%
  text={#3\glsadd{#2}},%
  description={{#5}},%
  #1
  }%
  \newglossaryentry{fsry:#2}{%
  type=\acronymtype,%
  first={#4 (#3)},%
  firstplural={#4\glspluralsuffix{} (\glsentryname{#2}\glspluralsuffix)},%
  name={#3\glsadd{fsry:main-#2}},%
  description={\glslink{fsry:main-#2}{#4}}%
  }%
}


